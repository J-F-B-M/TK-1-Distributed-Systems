\documentclass{tudexercise}

\usepackage[english]{babel}

\begin{document}
	\section{Threading}
		A Monitor is a block of data with some procedures, that can only be accessed by one process at any time. The procedures can range from simple Setters/Getters to complex calculations. Usually, Fields are not directly accessible, but are accessed via the mentioned Setters/Getters.  
		If two Processes try to access any monitored (in Java synchronized) method of the Monitor, the second process is halted until the first completes. This enables one to definitely say "one process executed before another", also known as \textit{happened-before-relationship}
		
	\section{Transparency in Java RMI}
		\begin{itemize}
			\item[Access] Yes, after some setup. Once remote objects are received and assigned, one can use them as any local object.
			\item[Location] Yes and no. No, because one still has to know where to fetch the remote objects from. Once the remote objects are fetched however, they can be accessed without any further location-knowledge about them.
			\item[Concurrency] Yes, because each procedure call will result in a new thread on the server. Of course the server providing the remote object may hold an internal state which can have problems with race conditions, but RMI itself is concurrent transparent.
			\item[Replication] Yes, the client asks the server to do something for him (procedure call), how the server handles this request is hidden from the client. 
			\item[Failure] No, RMI will tell the user and/or program that it encountered an error and will leave it to them to handle the problem.
			\item[Mobility] Yes, if the servers network address is not hardcoded but can be changed (for example with the help of a config-file).
			\item[Performance] No. The connection to the remote server always needs more time than a local call. However, if the machines performance is by magnitudes worse than the servers performance, this extra networking effort can still be cheaper than a local procedure call. For example: Render a complex 3D-Scene either on a small netbook or on a serverfarm. While the call of the procedure is fast on the netbook itself, the procedure is only fast on the serverfarm.
			\item[Scaling] Yes, if the underlying algorithm is scalable. As with all types of scalability this is of course a limited yes. Network connection limitations may slow the process flow so much that a proper execution is almost impossible. For small scales there should be however usually no problems.
		\end{itemize}
		
	\section{RMI - single-threaded vs. multi-threaded}
		\begin{itemize}
			\item[Single-Thread] The thread has to compute the arguments (total: 5ms), marshalls them (5.5ms), send them (5.7ms), transmit them (10.7ms), the server receives them (10.9ms), demarshalls them (11.4ms), processes them (21.4ms), marshalls the result (21.9ms), sends them (22.1ms), transmit them (27.1ms), the client receives the result (27.3ms) and demarshalls it (27.8ms). The whole thing happens again, for a total of \textbf{55.6 ms}.
			
			\item[Multi-Thread] Thread 1 computes, marshalls and sends his message (5.7ms). While the Message travels, Thread 2 starts his computation. The message arrives at the server at 10.7ms, the server starts his own process. The second thread finishes computing at 11.4ms and his message is on the way. At 16.4ms the message reaches the server, but the server is busy. The server finishes the first request (including sending) at 22.1ms, and immediatly starts the second request. At 27.1ms the first result reaches Thread 1, and at \textbf{27.8ms Thread 1 finishes}. The server finishes the second request 33.5ms, at 38.5ms it reaches the second Thread and at \textbf{39.2ms Thread 2 finishes}. The total process took \textbf{39.2ms}.
		\end{itemize}
\end{document}